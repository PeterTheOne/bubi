\documentclass[parskip=half,12pt,a4paper]{scrartcl}
\usepackage{ucs}
\usepackage[utf8x]{inputenc}
\usepackage[T1]{fontenc}
\usepackage[naustrian]{babel}
\usepackage{listings}
\usepackage{booktabs}
\usepackage{tabularx}
\usepackage[automark]{scrpage2}
\pagestyle{scrheadings}

\begin{document}
\title{Beispiel 4: Ablauf der Doppik - Unternehmensgründung Leitl}
\author{Dipl.-Ing. Herbert Mühlburger}
\maketitle

Im folgenden werden die
\begin{enumerate}
	\item Eröffnungsbuchungen 
	\item Verbuchung der laufenden Geschäftsfälle
	\item Nachbuchungen
	\item Abschlussbuchungen
\end{enumerate}
dargestellt.

\minisec{Eröffnungsbilanz zum 01.10.}
\begin{center}
	\begin{tabular}{lr|lr}
		\multicolumn{4}{c}{Eröffnungsbilanz 01.10.}\\
		\toprule
		Anlagevermögen & 0.- & Eigenkapital & 150.000.-\\
		Umlaufvermögen & 150.000.- & Fremdkapital & 0.-\\
		\bottomrule
		Summe & 150.000.- & Summe & 150.000.-\\
	\end{tabular}
\end{center}

\section{Eröffnungsbuchungen}

\minisec{(1) 01.10}
\begin{center}
\begin{tabularx}{\textwidth}{rXrr}
 \toprule
    & (2700) Kassa & 150.000.- &\\
 an & (9800) Eröffnungsbilanzkonto & & 150.000.-\\
 \bottomrule
\end{tabularx}
\end{center}

\minisec{(2) 01.10}
\begin{center}
\begin{tabularx}{\textwidth}{rXrr}
 \toprule
    & (9800) Eröffnungsbilanzkonto & 150.000.- &\\
 an & (9000) Eigenkapitalkonto (EKK) & & 150.000.-\\
\bottomrule
\end{tabularx}
\end{center}

\minisec{Eröffnungsbilanzkonto}
\begin{center}
	\begin{tabular}{lr|lr}
		\multicolumn{4}{c}{(9899) Eröffnungsbilanzkonto}\\
		\toprule
		Eigenkapital & 150.000.- & Kassa & 150.000.- \\
		\bottomrule
		Summe & 150.000.- & Summe & 150.000.-\\
	\end{tabular}
\end{center}

Damit sind die Eröffnungsbuchungen abgeschlossen. Alle weiteren Buchungen betreffen die laufenden Geschäftsfälle.

\section{Verbuchung der laufenden Geschäftsfälle}

\minisec{(3) 02.10}
\begin{center}
\begin{tabularx}{\textwidth}{rXrr}
 \toprule
    & (7400) Mietaufwand & 2.000.- &\\
 an & (2700) Kassa & & 2.000.-\\
\bottomrule
\end{tabularx}
\end{center}

\minisec{(4) 05.10}
\begin{center}
\begin{tabularx}{\textwidth}{rXrr}
 \toprule
    & (0600) Büroeinrichtungen & 7.000.- &\\
 an & (2700) Kassa & & 7.000.-\\
\bottomrule
\end{tabularx}
\end{center}

\minisec{(5) 08.10}
\begin{center}
\begin{tabularx}{\textwidth}{rXrr}
 \toprule
    & (1600) Waren (Handelswaren) & 80.000.- &\\
 an & (3300) Verbindlichkeiten aus Lieferungen u. Leistung Inland & & 80.000.-\\
\bottomrule
\end{tabularx}
\end{center}

\minisec{(6) 12.10}
\begin{center}
\begin{tabularx}{\textwidth}{rXrr}
 \toprule
    & (2000) Lieferforderungen & 100.000.- &\\
 an & (4000) Umsatzerlöse Inland (20\% USt) & & 100.000.-\\
\bottomrule
\end{tabularx}
\end{center}

\minisec{(7) 15.10}
\begin{center}
\begin{tabularx}{\textwidth}{rXrr}
 \toprule
    & (3300) Lieferverbindlichkeiten & 20.000.- &\\
 an & (2700) Kassa & & 20.000.-\\
\bottomrule
\end{tabularx}
\end{center}

\section{Nachbuchungen}

In diesem Beispiel werden die Abschreibung und der Verbrauch in den Nachbuchungen erfasst.

\minisec{(8) 31.10}
\begin{center}
\begin{tabularx}{\textwidth}{rXrr}
 \toprule
    & (7020) Planmäßige Abschreibung & 200.- &\\
 an & (0600) Büroeinrichtung & & 200.-\\
\bottomrule
\end{tabularx}
\end{center}

Die Inventur am 31.10. ergab einen Handelswarenvorrat von 100.-. Dadurch kann der Handelswarenverbrauch berechnet werden: Gekaufte Handeslwaren im Wert von 80.000.- minus dem Handelswarenvorrat am 31.10 von 100.- ergibt 79.900.-. Dieser Wert wird jetzt alsl Aufwand verbucht:

\minisec{(9) 31.10}
\begin{center}
\begin{tabularx}{\textwidth}{rXrr}
 \toprule
    & (5300) Handelswarenverbrauch & 79.900.- &\\
 an & (1600) Waren (Handelswaren) & & 79.900.-\\
\bottomrule
\end{tabularx}
\end{center}

\section{Abschlussbuchungen}
\subsection{G\&V Buchungen}

Im nächsten Schritt werden alle Erfolgskonten (Aufwands- und Ertragskonten) gegen das Gewinn und Verlustkonto (GuV-Konto) abgeschlossen. Das GuV-Konto wird am Ende gegen das Eigenkapitalkonto (EKK) abgeschlossen.

\minisec{(10) 31.10}
\begin{center}
\begin{tabularx}{\textwidth}{rXrr}
 \toprule
    & (9890) GuV & 2.000.- &\\
 an & (7400) Mietaufwand & & 2.000.-\\
\bottomrule
\end{tabularx}
\end{center}

\minisec{(11) 31.10}
\begin{center}
\begin{tabularx}{\textwidth}{rXrr}
 \toprule
    & (4000) Umsatzerlöse Inland (20\% USt) & 100.000.- &\\
 an & (9890) GuV & & 100.000.-\\
\bottomrule
\end{tabularx}
\end{center}

\minisec{(12) 31.10}
\begin{center}
\begin{tabularx}{\textwidth}{rXrr}
 \toprule
    & (9890) GuV & 200.- &\\
 an & (7020) Planmäßige Abschreibung & & 200.-\\
\bottomrule
\end{tabularx}
\end{center}

\minisec{(13) 31.10}
\begin{center}
\begin{tabularx}{\textwidth}{rXrr}
 \toprule
    & (9890) GuV & 79.900.- &\\
 an & (5300) Handelswarenverbrauch & & 79.900.-\\
\bottomrule
\end{tabularx}
\end{center}

\minisec{(14) 31.10}
\begin{center}
\begin{tabularx}{\textwidth}{rXrr}
 \toprule
    & (9890) GuV & 17.900.- &\\
 an & (9000) Eigenkapitalkonto & &17.900.-\\
\bottomrule
\end{tabularx}
\end{center}

\subsection{Schlussbilanzbuchungen}
Im nächsten Schritt werden alle Bestandskonten (aktive und passive) gegen das Schlussbilanzkonto abgeschlossen.

\minisec{(15) 31.10}
\begin{center}
\begin{tabularx}{\textwidth}{rXrr}
 \toprule
    & (9850) SBK & 6.800.- &\\
 an & (0600) Büroeinrichtungen & & 6.800.-\\
\bottomrule
\end{tabularx}
\end{center}

\minisec{(16) 31.10}
\begin{center}
\begin{tabularx}{\textwidth}{rXrr}
 \toprule
    & (9850) SBK & 100.- &\\
 an & (1600) Waren (Handelswaren) & & 100.-\\
\bottomrule
\end{tabularx}
\end{center}

\minisec{(17) 31.10}
\begin{center}
\begin{tabularx}{\textwidth}{rXrr}
 \toprule
    & (9850) SBK & 121.000.- &\\
 an & (2700) Kassa & & 121.000.-\\
\bottomrule
\end{tabularx}
\end{center}

\minisec{(18) 31.10}
\begin{center}
\begin{tabularx}{\textwidth}{rXrr}
 \toprule
    & (9850) SBK & 100.000.- &\\
 an & (2000) Lieferforderungen & & 100.000.-\\
\bottomrule
\end{tabularx}
\end{center}

\minisec{(19) 31.10}
\begin{center}
\begin{tabularx}{\textwidth}{rXrr}
 \toprule
    & (3300) Lieferverbindlichkeiten & 60.000.- &\\
 an & (9850) SBK & & 60.000.-\\
\bottomrule
\end{tabularx}
\end{center}

\minisec{(20) 31.10}
\begin{center}
\begin{tabularx}{\textwidth}{rXrr}
 \toprule
    & (9000) Eigenkapitalkonto & 167.900.- &\\
 an & (9850) SBK & & 167.900.-\\
\bottomrule
\end{tabularx}
\end{center}

\section{Hauptbuch (dargestellt als T-Konten)}

\subsection{aktive Bestandskonten}

\begin{center}
	\begin{tabular}{lr|lr}
		\multicolumn{4}{c}{(0600) Büroeinrichtungen}\\
		\toprule
		(4) & 7.000.- & (8) & 200.-\\
		& & (15) & 6.800.-\\
		\bottomrule
	\end{tabular}
\end{center}

\begin{center}
	\begin{tabular}{lr|lr}
		\multicolumn{4}{c}{(1600) Waren (Handelswaren)}\\
		\toprule
		(5) & 80.000.- & (9) & 79.900.-\\
		& & (16) & 100.-\\
		\bottomrule
	\end{tabular}
\end{center}

\begin{center}
	\begin{tabular}{lr|lr}
		\multicolumn{4}{c}{(2000) Lieferforderungen}\\
		\toprule
		(6) & 100.000.- & (18) & 100.000.-\\
		\bottomrule
	\end{tabular}
\end{center}

\begin{center}
	\begin{tabular}{lr|lr}
		\multicolumn{4}{c}{(2700) Kassa}\\
		\toprule
		(1) & 150.000.- & (3) & 2.000.-\\
		& & (4) & 7.000.-\\
		& & (7) & 20.000.-\\
		& & (17) & 121.000.-\\
		\bottomrule
	\end{tabular}
\end{center}



\subsection{passive Bestandskonten}

\begin{center}
	\begin{tabular}{lr|lr}
		\multicolumn{4}{c}{(3300) Lieferverbindlichkeiten}\\
		\toprule
		(7) & 20.000.- & (5) & 80.000.-\\
		(19) & 60.000.- & &\\
		\bottomrule
	\end{tabular}
\end{center}

\subsection{Ertragskonten}

\begin{center}
	\begin{tabular}{lr|lr}
		\multicolumn{4}{c}{(4000) Umsatzerlöse Inland (20\% USt)}\\
		\midrule
		(11) & 100.000.- & (6) & 100.000.-\\
		\bottomrule
	\end{tabular}
\end{center}

\subsection{Aufwandskonten}

\minisec{Warenverbrauch}
\begin{center}
	\begin{tabular}{lr|lr}
		\multicolumn{4}{c}{(5300) Handelswarenverbrauch}\\
		\toprule
		(9) & 79.900.- & (13) & 79.900.-\\
		\bottomrule
	\end{tabular}
\end{center}

\begin{center}
	\begin{tabular}{lr|lr}
		\multicolumn{4}{c}{(7020) Planmäßige Abschreibung}\\
		\toprule
		(8) & 200.- & (12) & 200.-\\
		\bottomrule
	\end{tabular}
\end{center}

\begin{center}
	\begin{tabular}{lr|lr}
		\multicolumn{4}{c}{(7400) Mietaufwand}\\
		\toprule
		(3) & 2.000.- & (10) & 2.000.-\\
		\bottomrule
	\end{tabular}
\end{center}

\subsection{Hilfskonten}

\begin{center}
\begin{tabular}{lr|lr}
	\multicolumn{4}{c}{(9000) Eigenkapitalkonto}\\
	\toprule
	(20) & 167.900.- & (2) & 150.000.-\\
	& & (14) & 17.900.-\\
	\bottomrule
\end{tabular}
\end{center}

\begin{center}
\begin{tabular}{lr|lr}
\multicolumn{4}{c}{(9800) Eröffnungsbilanzkonto}\\
\toprule
(2) & 150.000.- & (1) & 150.000.-\\
\end{tabular}
\end{center}

\begin{center}
\begin{tabular}{lr|lr}
\multicolumn{4}{c}{(9850) SBK}\\
\toprule
(15) & 6.800.- & (19) & 60.000.-\\
(16) & 100.- & (20) & 167.900.-\\
(17) & 121.000.- & &\\
(18) & 100.000.- & &\\
\bottomrule
\end{tabular}
\end{center}

\begin{center}
	\begin{tabular}{lr|lr}
		\multicolumn{4}{c}{(9890) GuV}\\
		\toprule
		(10) & 2.000.- & (11) & 100.000.-\\
		(12) & 200.- & &\\
		(13) & 79.900.- & &\\
		(14) & 17.900.- & &\\
		\bottomrule
	\end{tabular}
\end{center}

Die Gewinnermittlung über den Reinvermögensvergleich wird folgendermaßen durchgeführt:

\minisec{Eröffnungsbilanz}
\begin{center}
	\begin{tabular}{lr|lr}
		\multicolumn{4}{c}{Eröffnungsbilanz 01.10.}\\
		\toprule
		Anlagevermögen & 0.- & Eigenkapital & 150.000.-\\
		Umlaufvermögen & 150.000.- & Fremdkapital & 0.-\\
		\bottomrule
		Summe & 150.000.- & Summe & 150.000.-\\
	\end{tabular}
\end{center}

\minisec{Schlussbilanz}
\begin{center}
	\begin{tabular}{lr|lr}
		\multicolumn{4}{c}{Schlussbilanz 31.10.}\\
		\toprule
		Anlagevermögen & 6.800.- & Eigenkapital & 167.900.-\\
		Umlaufvermögen & 221.100.- & Fremdkapital & 60.000.-\\
		\bottomrule
		Summe & 227.900.- & Summe & 227.900.-\\
	\end{tabular}
\end{center}

Die Ermittlung des Periodengewinnes durch Betriebsvermögensvergleich wird folgendermaßen durchgeführt:

\begin{center}
	\begin{tabular}{lcrr}
		Gesamtvermögen & 31.10. & 227.900.- &\\
		- Fremdkapital & 31.10. & -60.000.- &\\
		\midrule
		Reinvermögen & 31.10 & 167.900.-& 167.900.-\\		
		Gesamtvermögen & 01.10. & 150.000.- &\\
		- Fremdkapital & 01.10. & -0.- &\\
		\midrule
		Reinvermögen & 01.10. & 150.000.- & -150.000.-\\
		\midrule
		Gewinn & & & 17.900.-\\
		\bottomrule
	\end{tabular}
\end{center}

Somit ergibt sich ein Gewinn von Euro \emph{17.900.-}.

\end{document}