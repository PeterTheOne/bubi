\documentclass[parskip=half,12pt,a4paper]{scrartcl}
\usepackage{ucs}
\usepackage[utf8x]{inputenc}
\usepackage[T1]{fontenc}
\usepackage[naustrian]{babel}
\usepackage{listings}
\usepackage{booktabs}
\usepackage{tabularx}
\usepackage[automark]{scrpage2}
\pagestyle{scrheadings}

\begin{document}

\title{Beispiel 6: Beschaffung und Absatz}
\author{Dipl.-Ing. Herbert Mühlburger}
\maketitle

Bei diesem Beispiel sind die Käufersicht und Verkäufersicht zu berücksichtigen. Zudem werden Vorsteuer und Umsatzsteuer jetzt auch mitberücksichtigt.

\section{Sicht des Käufers}

\minisec{(1) 10.10. Einkauf}
\begin{center}
	\begin{tabularx}{\textwidth}{rXrr}
		\toprule
		& (1600) Waren (Handeslwaren) & 640.- &\\
		an & (3300) Verbindlichkeiten & & 640.-\\
		\bottomrule
	\end{tabularx}
\end{center}

\minisec{(2) 10.10. Einkauf}
\begin{center}
	\begin{tabularx}{\textwidth}{rXrr}
		\toprule
		& (2500) Vorsteuer & 64.- &\\
		an & (3300) Verbindlichkeiten & & 64.-\\
		\bottomrule
	\end{tabularx}
\end{center}

Diese beiden Buchungssätze können auch in einem zusammengefasst werden:

\minisec{(1) 10.10. Einkauf}
\begin{center}
	\begin{tabularx}{\textwidth}{rXrr}
		\toprule
		& (1600) Waren (Handeslwaren) & 640.- &\\
		& (2500) Vorsteuer & 64.- &\\
		an & (3300) Verbindlichkeiten & & 704.-\\
		\bottomrule
	\end{tabularx}
\end{center}

Die Bezahlung erfolgt, wie im Beispiel angegeben am 17.10.:

\minisec{(3) 17.10. Überweisung}
\begin{center}
	\begin{tabularx}{\textwidth}{rXrr}
		\toprule
		& (3300) Verbindlichkeiten & 704.- &\\
		an & (2800) Bank & & 704.-\\
		\bottomrule
	\end{tabularx}
\end{center}

\section{Sicht des Verkäufers}

\minisec{(1) 10.10. Verkauf (netto)}
\begin{center}
	\begin{tabularx}{\textwidth}{rXrr}
		\toprule
		& (2000) Forderungen & 640.- &\\
		an & (4000) Umsatzerlöse & & 640.- \\
		\bottomrule
	\end{tabularx}
\end{center}

\minisec{(1) 10.10. Verkauf (Umsatzsteuer)}
\begin{center}
	\begin{tabularx}{\textwidth}{rXrr}
		\toprule
		& (2000) Forderungen & 64.- &\\
		an & (3300) Umsatzsteuer & & 64.-\\
		\bottomrule
	\end{tabularx}
\end{center}

Am 19.10. erscheint das Guthaben am Tagesauszug der Fleischerei.

\minisec{(3) 19.10. Einzahlung}
\begin{center}
	\begin{tabularx}{\textwidth}{rXrr}
		\toprule
		& (2800) Bank & 704.- &\\
		an & (2000) Forderungen & & 704.- \\
		\bottomrule
	\end{tabularx}
\end{center}

\end{document}